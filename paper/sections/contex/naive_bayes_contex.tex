\subsubsection{BI-LSTM}

\noindent
Naïve Bayes is a simple and quick probabilistic classifier that uses Bayes' theorem, assuming independence between features. It works well in text classification, especially for high-dimensional data and sparse input, which is common in natural language processing (NLP). Naïve Bayes calculates the probability of a class based on specific terms, effectively detecting hate speech by analyzing occurrences of certain words.

Recent studies, including "Hate Speech Detection in Social Networks using Machine Learning and Deep Learning Methods," highlight Naïve Bayes' effectiveness on platforms like Twitter, achieving a precision of 0. 832 and a recall of 0. 863, resulting in an F1-score of 0. 851. Another study by Asif et al. (2024), while noting Naïve Bayes' limitations in capturing context, recognizes its computational efficiency and acceptable recall, making it a practical choice compared to deep learning models.