\subsubsection{XGBoost}

One of the models that drew the most attention when choosing which to use for the toxicity classification task was XGBoostClassifier. According to \cite{xgboost2016}, XGBoost stands for Extreme Gradient Boosting and is based on the use of multiple decision and regression trees to create an ensemble method with greater predictive power. In this model, scores are assigned to examples according to the leaf in which they are classified by each tree, and then those scores are summed. This approach is similar to that used in Random Forests, with the difference that training is performed through Boosting. The goal of the method is to learn both the structure and the scores of the trees. In each training iteration, it seeks to find the tree that contributes the most to optimizing the objective, which can vary depending on the task. It can be said that the trees in the ensemble method complement each other to solve the task for which the training is performed.

In \cite{comparison2023toxic}, the ensemble methods tested were AdaBoost and Random Forest, achieving an accuracy of 95.518 using Random Forest. On the other hand, in \cite{fieri2021soft}, using this same classifier, an accuracy of 91.88 was reached. Finally, in \cite{bonetti2021hate}, also with this classifier, an accuracy of 85.1 was obtained. Based on these results, these scores can be taken as a reference and the search for some improvement can be established.


