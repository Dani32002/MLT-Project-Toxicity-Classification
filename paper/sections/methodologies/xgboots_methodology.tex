\subsubsection{Methodology XGBoost}

The methodology used for the XGBoostClassifier began by loading the [CLS] token embeddings generated by RoBERTa, that is, the vector that summarizes the entire text sequence. After this, the data was split into 80\% for training and 20\% for testing. This split resulted in 38,152 examples available for training. Subsequently, a parameter grid was defined to conduct hyperparameter tuning. The search focused on the number of estimators, maximum tree depth, and minimum child weight. To find the best values, a grid search with 3-fold cross-validation was used.

In addition to this search, other hyperparameters were set manually. For example, 80\% of the data was used in each boosting round, and in each tree and node, a sample of features was selected such that its size was proportional to the square root of the total number of features, as recommended in \cite{Chen2016XGBoost}.

For training, 50\% of the dataset was used to perform the hyperparameter search, as doing it on the entire training set would have been computationally too expensive. Once the best hyperparameters were found, the model was trained on the full training set and then evaluated.

To evaluate the model, a classification report was generated that measured precision, recall, F1-score, and accuracy. Additionally, a confusion matrix was presented to more clearly observe the trend of the data toward true positives or misclassification.
