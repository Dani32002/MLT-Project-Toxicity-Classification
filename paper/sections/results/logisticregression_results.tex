\subsection{Random Forest Results}
\label{sec:random_forest_results}

As part of the machine learning baseline models evaluated in this study, a Random Forest classifier was implemented and trained on the preprocessed dataset. This ensemble method, which operates by constructing multiple decision trees and aggregating their outputs, is known for its robustness and resistance to overfitting, especially in high-dimensional textual data. The model yielded promising results on the test set, achieving an \textbf{accuracy} of 0.8627, a \textbf{precision} of 0.8980, a \textbf{recall} of 0.8246, and an \textbf{F1-score} of 0.8597. These metrics indicate a balanced performance in detecting both hate and non-hate speech categories, demonstrating the classifier's effectiveness even in comparison with more complex deep learning approaches. In terms of interpretability and computational efficiency, Random Forest remains a strong candidate for practical deployment in moderation systems.
