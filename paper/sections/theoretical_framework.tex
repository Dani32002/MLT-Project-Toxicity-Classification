This study presents a comprehensive methodological framework for the detection of hate speech on Twitter, integrating both machine learning and deep learning approaches. The process began with an extensive data preprocessing pipeline to ensure data consistency, clean noisy text, and address class imbalance using Easy Data Augmentation (EDA)~\cite{wei2019eda}. This included standard text normalization techniques—such as stopword removal and stemming—as well as more advanced steps like automated spelling correction and synonym expansion.

Subsequently, multiple classification models were implemented and evaluated. For traditional machine learning, models such as Support Vector Machines (SVM), Logistic Regression, and Multinomial Naive Bayes were applied, providing baseline performances and interpretability advantages. On the deep learning side, we implemented architectures including Convolutional Neural Networks (CNN) and Gated Recurrent Units (GRU), both trained on pretrained embeddings to capture contextual linguistic features.

In addition, transformer-based language models were explored, notably RoBERTa, for which both zero-shot evaluation and supervised fine-tuning were performed. The fine-tuning involved training a classification head on top of RoBERTa’s pretrained layers using our balanced dataset.

This comparative methodology allows for a fair assessment across heterogeneous architectures, evaluating their strengths and limitations under a consistent experimental setup inspired by previous works~\cite{fieri2023offensive,almeida2023comparison}.

\subsection{Machine learning Methods}
This section describes the classification methods used in this study. Each represents a different approach within supervised learning.

\subsubsection{Logistic Regression}
\textbf{Definition}
Logistic regression is a statistical model used to predict the probability of a binary class. It is efficient for linear problems and serves as a baseline for comparison with more complex models.

To reference one of the earliest formulations of the model, we draw upon the perspective presented in \textit{The Regression Analysis of Binary Sequences} by D. R. Cox (1958), which introduced a rigorous statistical framework for the analysis of binary outcomes. In this work, both dependent and independent variables are explicitly identified, and the probability of a binary event is modeled using a logistic transformation of the predictors. Dependent variables are represented as binary categorical variables, typically coded as 0 and 1, which enables the analysis of how one or more explanatory variables influence the likelihood of an event of interest. Through illustrative examples, the author demonstrates how the model can be estimated using maximum likelihood methods, thereby establishing a methodological foundation that continues to be widely applied in fields such as biostatistics, the social sciences, and machine learning \citep{cox1958logistic}.

Given the capabilities offered by the model, it has been selected for use in this study. However, prior to its implementation, we will review some previous approaches used in earlier research.

As one of the initial and relevant approaches, the study presented in the article "Offensive Language Detection Using Soft Voting Ensemble Model" \citep{supert2023offensive} incorporates logistic regression as part of its ensemble strategy. The article emphasizes the use of L2 regularization, which enhances the precision of offensive language classification by mitigating overfitting. Furthermore, the model's simplicity is highlighted as a significant advantage, enabling efficient implementation even with limited computational resources—demonstrating its usefulness and versatility within the set of evaluated classifiers.

The research article titled "Comparison between Machine Learning and Deep Learning Approaches for the Detection of Toxic Comments on Social Networks" \citep{bonetti2023comparison} employs logistic regression once again, combining it with two semantic modeling techniques: Latent Semantic Analysis (LSA) and Latent Dirichlet Allocation (LDA). These methods will be further discussed in the following paragraphs. Notably, the study reports a prediction accuracy exceeding 91\%, highlighting the competitiveness of traditional approaches compared to more complex deep learning models.

\subsubsection{XGBoost}

One of the models that drew the most attention when choosing which to use for the toxicity classification task was XGBoostClassifier. According to \cite{xgboost2016}, XGBoost stands for Extreme Gradient Boosting and is based on the use of multiple decision and regression trees to create an ensemble method with greater predictive power. In this model, scores are assigned to examples according to the leaf in which they are classified by each tree, and then those scores are summed. This approach is similar to that used in Random Forests, with the difference that training is performed through Boosting. The goal of the method is to learn both the structure and the scores of the trees. In each training iteration, it seeks to find the tree that contributes the most to optimizing the objective, which can vary depending on the task. It can be said that the trees in the ensemble method complement each other to solve the task for which the training is performed.

In \cite{fieri2023offensive}, the ensemble methods tested were AdaBoost and Random Forest, achieving an accuracy of 95.518 using Random Forest. On the other hand, in \cite{bonetti2023comparison}, using this same classifier, an accuracy of 91.88 was reached. Finally, in \cite{toktarova2023hate}, also with this classifier, an accuracy of 85.1 was obtained. Based on these results, these scores can be taken as a reference and the search for some improvement can be established.

\subsubsection{Na\"ive Bayes}

\noindent
Naïve Bayes is a simple and quick probabilistic classifier that uses Bayes' theorem, assuming independence between features. It works well in text classification, especially for high-dimensional data and sparse input, which is common in natural language processing (NLP). Naïve Bayes calculates the probability of a class based on specific terms, effectively detecting hate speech by analyzing occurrences of certain words.

Recent studies, including Hate Speech Detection in Social Networks using Machine Learning and Deep Learning Methods \citep{hate2022bonetti} highlight Naïve Bayes' effectiveness on platforms like Twitter, achieving a precision of 0. 832 and a recall of 0. 863, resulting in an F1-score of 0. 851. Another study by Asif et al. (2024), while noting Naïve Bayes' limitations in capturing context, recognizes its computational efficiency and acceptable recall, making it a practical choice compared to deep learning models.
\subsubsection{SVM (Support Vector Machines)}

SVM (Support Vector Machines) is a machine learning method mainly used for classification tasks. Its main goal is to find the best hyperplane that separates the classes and maximizes the margin with the closest points from each class. This process involves identifying the support vectors, which are the closest points to the classification hyperplane. A common technique used with SVM is the use of kernels. With kernels, data points are transformed into a new space that might be linearly separable, improving classification results. There is a variety of kernels with different transformation approaches to choose from. An important hyperparameter for this algorithm is “C,” which refers to a regularization term that controls the trade-off between margin maximization and misclassifications~\cite{geeksforgeeks_svm}.


\subsection{Deep Learning Methods}
Deep learning is a subset of machine learning that uses artificial neural networks with many layers to automatically learn complex patterns from large amounts of data.

\subsubsection{BI-LSTM}

\noindent
\cite{toktarova2023hate} BI-LSTM (Bidirectional Long Short-Term Memory) is a type of sequential neural network that allows working with data composed of different elements ordered in a sequence, such as text. This type of network can make predictions about parts of the sequence based on what has been seen so far or on the entire input. It is an improvement over LSTM networks, which, like recurrent networks, use output values from previous steps in the sequence. However, it differs by dividing the network's responsibilities into different flow control gates. Additionally, it is capable of maintaining both long-term memory and a hidden state, which allows it to respond appropriately to the current step in the sequence. BI-LSTMs process the input both forward and backward, thus capturing future and past context at each step. In \cite{fieri2021soft}, using BI-LSTM, an accuracy of 90.2 was achieved; meanwhile, in \cite{bonetti2023comparison}, 96.102 was reached.


\subsubsection{roBERTa}
RoBERTa is a bidirectional pretrained encoder based on BERT, but trained on a larger amount of data and with longer sequences. BERT itself is based on the encoder component of the Transformer architecture, which generates vector representations by considering both the past and future context of the sequence elements. The pretrained knowledge of the model can be reused for other tasks by simply adding a classification head and fine-tuning the pretrained weights to solve the new problem. In \cite{bonetti2023comparison}, BERTweet was used — a model similar to RoBERTa but specifically trained on Twitter data — achieving an accuracy of 92.38.
\subsubsection{GRU}

\noindent
Gated recurrent Units (GRUs) are part of the recurrent neural network (RNNs) designed to work with sequential data. GRUs use a gating mechanisms with update and reset gates to control flow information over time, which allows the model to remember or forget information effectively. This helps to capture relationships in text while having a less complex structure like for example LSTM networks.

GRUs have shown strong performance when in use for hate speech detection in social media. In a Comparison between Machine Learning and Deep Learning Approaches for the Detection of Toxic Comments on Social Networks \citep{bonetti2023comparison}, GRUs are compared with LSTM and CNN models, in this case GRUs performed well in accuracy and adaptability, especially with diverse and informal language. Another study Hate Speech Detection in Social Networks using Machine Learning and Deep Learning Methods \citep{faisal2023hate} highlighted GRUs efficiency in recognizing patterns in toxic comments, achieving a high F1-score of 0.907 compared to traditional methods. GRUs are efficient for real-time content moderation, balancing performance and speed.

\subsubsection{Convolutional Neural Networks (CNNs)}
In the reviewed studies~\cite{bonetti2023comparison,fieri2023offensive}, Convolutional Neural Networks (CNNs) have been employed as an efficient deep learning approach for the detection of offensive language and toxic comments. Their ability to extract spatial patterns from text embeddings enables the identification of relevant structures within input sequences, leading to effective classification. However, although CNNs exhibit competitive performance, the studies also emphasize that their effectiveness may be surpassed by models capable of capturing full-sequence context, such as bidirectional recurrent neural networks or transformer-based models, particularly when dealing with complex or ambiguous natural language inputs.

\subsubsection{MLP}
An MLP (Multi-Layer Perceptron) is the simplest and most fundamental form of neural networks and deep learning. It builds on the concept of neurons, grouping them into layers and then stacking those layers. Each neuron performs a simple weighted sum, but when organized in layers, the input data can be transformed into a new feature space. MLP architectures typically consist of an input layer, multiple hidden layers, and an output layer, which is responsible for producing the final result. These networks are usually fully connected, meaning that every output from one layer is passed to each node in the next layer. Layers are paired with activation functions that introduce non-linearity and help transform outputs into a desired range, such as between 0 and 1 to represent a probability. MLPs are trained using an ``error'' metric, such as cross-entropy loss, and optimized using algorithms like gradient descent through backpropagation across all weights~\cite{mlp_gfg}.
